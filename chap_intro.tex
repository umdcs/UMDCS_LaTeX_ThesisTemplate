\chapter{Introduction}
\label{chap:intro}

It's always good to introduce your (1) problem, (2) why it is interesting, (3) what you did, and (4) roughly, how well did it work. You might even have citations in here, as in this paper~\cite{Asawa:2008:TDT}.  It is helpful to think about the introduction as both a way to motivate the problem you are working on and convince the the reader why that problem should be worked on and that they should read further. Including brief summaries of your results or what you did to achieve your results are also encouraged as these can help the reader understand where your work is going as they read more.
