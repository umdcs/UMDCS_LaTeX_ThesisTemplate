% M.S. Computer Science Thesis
\documentclass{umdthesis}

% This package will nag about old-style LaTeX use.  Feel free to
% uncomment if you care about such issues.  \usepackage[l2tabu,
% orthodox]{nag}

% Note as of 11/18/14
% At the time of this update, the website:
% http://www.olivierverdier.com/posts/2013/07/15/modern-latex/ had
% useful information for Modern LaTeX... some of it is included here.

% Uses new biblatex system rather than traditional bibtex
% This uses the Author-Year style for both citations and references
\usepackage[style=authoryear, citestyle=authoryear, backref=true, isbn=true, url=true, backend=biber]{biblatex}

% There doesn't seem to much different if you use the apa style in the same way
\usepackage[style=apa, citestyle=apa, backref=true, isbn=true, url=true, backend=biber]{biblatex}

% allows the Table of Contents to include the figure listings.
\usepackage{tocbibind}
% changes the name from Bibliography to References which is more appropriate for CS
\DefineBibliographyStrings{english}{%
  bibliography = {References},
}

% Your references are placed into the .bib file and then specified here:
\addbibresource{UMDCS_MSThesis.bib}

% \usepackage{fontspec}
% \setmainfont{TimesNewRomanPSMT}
% \setmathfont{CambriaMath}
% \setmainfont{Times}
% \setsansfont{Helvetica}

% This will highlight code examples with nice coloring if you have that in your thesis.
% 
% This example uses C++, but there are many nice examples for other languages too.
%
% Nice C++ listings example derived from http://timmurphy.org/2014/01/27/displaying-code-in-latex-documents/
\definecolor{listinggray}{gray}{0.9}
\definecolor{lbcolor}{rgb}{0.95,0.95,0.95}

\lstset{
  backgroundcolor=\color{lbcolor},
  language=C++, % if you need to change the language...
  frame=lines, % draw a frame at the top and bottom of the code block
  tabsize=3, % tab space width
  captionpos=b,
  basicstyle=\footnotesize,
  breaklines=true,
  showstringspaces=false, % don't mark spaces in strings
  numbers=left, % display line numbers on the left
%  commentstyle=\color{green}, % comment color
  keywordstyle=\color{blue}, % keyword color
  stringstyle=\color{red} % string color
}

\RequirePackage{amsmath}
\RequirePackage{amssymb} 

% Supplied by Matt Overby - Spring 2014 - This is a nice command that
% lets you mark sections you will come back to later.
\newcommand{\TODO}[1]{\colorbox{yellow}{\textbf{TODO}: #1}}

\title{Your Great Title For Your Thesis}
\author{Pete Willemsen}

\advisor{Professor Advisor Name}  

\copyrightyear{2017} 

\ackfile{acknowledge} 
\dedication{dedication}
\abstractfile{abstract}

\begin{document} 
 
	\frontmatter 
 
        % You could save some space when initially printing (if you
	% really need to) by uncommenting this line to single
	% space
        %
        % \singlespace
      
        % Ideally the Introduction to your thesis 
	\chapter{Introduction}
\label{chap:intro}

It's always good to introduce your (1) problem, (2) why it is interesting, (3) what you did, and (4) roughly, how well did it work. You might even have citations in here, as in this paper~\cite{Asawa:2008:TDT}.  It is helpful to think about the introduction as both a way to motivate the problem you are working on and convince the the reader why that problem should be worked on and that they should read further. Including brief summaries of your results or what you did to achieve your results are also encouraged as these can help the reader understand where your work is going as they read more.
 
	 
        % Any background necessary to understand your thesis.  This 
        % can also contain the related work too. 
	\chapter{Background}
\label{chap:background}

\section{Background}

In this work, we investigate human computer interaction. 

\subsection{Robot Interfaces}

In seminar today, we looked for papers on the ACM Digital Library. The
following paper is about virtual reality \parencite{Kreylos:2006:ESW:1128923.1128948}, whereas this paper investigates robotics issues \parencite{Drascic89}.

\section{Previous Work}

Previous work by Ranga in this area focused on measurement of peer to peer sysmtes and found that .... ~cite{ranga15}. This thesis will build upon Ranga's work by exploiring t......

A new article showed that ultization is.... \parencite{Abbasi:2013:DBS:2507924.2507961}. Deep learning is everywhere \parencite{Tang:2019:MTR:3365594.3358696}.

\cite{website1}

\footnote{This information found on www.resilio.com on Oct 20, 2016.}

In seminar today, we looked for papers on the ACM Digital Library. The
following paper is about virtual reality
\parencite{Kreylos:2006:ESW:1128923.1128948}.

Found another paper... think the tile has VRGP in it... make sure to read.





 
 
        % Of course, if you wanted a separate chapter for Related
        % Work, you could make one too. 
	% \include{chap_relatedwork}     

        % The major content for your thesis!  What did you do and how   
        % did you do it!
	\chapter{Implementation}
\label{chap:impl}

\section{First section}

You may need a nice figure, which you can algorithmically render using the Tikz package. You should really check out the Texample web site where several nice tikz examples are provided (http://www.texample.net/tikz/examples/all/).

% this is how you include figures and this happens to include a tikzpicture which
% is a programming language for generating diagrams and other useful items.
\begin{figure}
\centering
\begin{tikzpicture}

\def \n {5}
\def \radius {3cm}
\def \margin {8} % margin in angles, depends on the radius

\foreach \s in {1,...,\n}
{
  \node[draw, circle] at ({360/\n * (\s - 1)}:\radius) {$\s$};
  \draw[->, >=latex] ({360/\n * (\s - 1)+\margin}:\radius) 
    arc ({360/\n * (\s - 1)+\margin}:{360/\n * (\s)-\margin}:\radius);
}
\end{tikzpicture}
\caption{Clear and concise figure captions are important to write. This one illustrates the cycle of a graph.}
\label{fig:tikzexample}
\end{figure}

\section{Initial Section}

Itemized lists are often used:
\begin{itemize}
\item What language the project was coded in?
\item What is the question the urban planners want solved?
\end{itemize}
 
Math is really nice with \LaTeX too!  Wonder what the following equation can say about the world?

\[L_{s}(\vec{k_{o}}) = L_{e}(\vec{k_{o}}) \int_{\text{all $\vec{k_{i}}$}} \rho(\vec{k}_i,\vec{k}_o) L_f(\vec{k}_i) \cos\theta_i d\sigma_i \]

Enumerated lists are helpful too
\begin{enumerate}
\item Step 1
\item Step 2
\item Step 3
\begin{itemize}
\item Important part of Step 3
\item Another important part of Step 3
\end{itemize}
\end{enumerate}

A graph rendered with the Tikz package is shown in \autoref{fig:tikzexample}.

\subsection{Subsection One}
\subsection{Subsection Two}
\subsubsection{Here's a Sub-Sub-Section}
\subsubsection{Here's another Sub-Sub-Section}
\subsection{Subsection Three}

\section{New Section For Next Important Topic}

\subsection{Algorithm Initialization}
\subsection{Atomic Operations}

You may even need code in your thesis. Here is a way to nicely include code with the \textit{listings} package in \LaTeX:
{\singlespace
\begin{lstlisting}
for (unsigned int idx=0; idx<maxSize; idx++) {
  atomic_add( idx );
}
\end{lstlisting}
}
If you need to modify some of the style of the listings package for this template, you may want to check the class style file included with this template: \textit{umdthesis.cls}.

\subsection{Programming Style}
\subsubsection{Explaining Fine Detail Here}

\TODO{Make sure to finish this description!}

\subsubsection{Last Subsection}
  

        % It's always good to have a results chapter so you can 
        % present how well your ideas and implementations worked
	\chapter{Results}
\label{chap:results}

Your results.  The idea of my thesis worked out great.  Here's a plot to show how great it worked.  

Sometimes, the TODO macro will be helpful for you to mark places where you still need to edit, or work on the document.  You'll want to remove the TODOs, of course, before submitting your thesis.
\TODO{Need to get results!!!! Make sure to finish this!}

% another figure, and this time we include an image from the images folder in this
% project.
\begin{figure}
\centering
\includegraphics[width=0.8\linewidth]{images/goodData.png}
\caption{Good data.}
\label{fig:goodData}
\end{figure} 

We can reference the plot in Figure~\ref{fig:goodData}. Also, it's sometimes nice to include tables.

% tables are often useful!
\begin{table}
\begin{center}
  \begin{tabular}{ | l | l | l | }
    \hline
    Variable & Condition 1 & Condition 2 \\ \hline
    \(arc\) & 1.796 & 0.304 \\ \hline 
    \(boo\) & 3.112 & 0.411 \\ \hline 
    \(gar\) & 4.344 & 0.629 \\ 
    \hline
  \end{tabular}
\end{center}
\caption{Illustrates the relationship between variables and the related experiment conditions.}
\label{tid:dataCond12}
\end{table}


        % Time to wrap up the thesis with a discussion of your ideas
        % and knowledge that you generated, along with any important
        % insights, or things you learned.  You can include ideas for
        % future extensions and effort here too.
	\include{chap_conclusion}
	
        % If you need an appendix (or appendices), they can be added here 
	\appendix
\chapter{Appendix A}

Do you need an Appendix?  You can include several of them if you want.  Just add more "chapters" with the Appendix Letter title.

	 
	
        % And finally, don't forget the references and bibliography.
        % You can add entries to the file UMDCS_Thesis.bib for your
        % references.  You then need to ``cite'' them in the tex files
        % by using the ~\cite{ReferenceID} tags.
        %
        % and make sure to include the bib in the TOC
	\newpage
        \printbibliography[heading=bibintoc]
	
\end{document}
